% !TEX root = ./main.tex
%%%%%%%%%%%%%%%%%%%%%%%%%%%%%%%%%%%%%%%%%%%%%%%%%%%%%%%%%%%%%%%%%%%%%%%%%%%%%%%%%%%%%%%%%%
%%%%%%%%%%%%%%%%%%%%%%%%%%%%%%%%%%%%%%%%%%%%%%%%%%%%%%%%%%%%%%%%%%%%%%%%%%%%%%%%%%%%%%%%%%
\section{Introduction}\label{introduction}
L'objectif de ce projet est de créer une fonction qui prend en entrées un tableau, sa taille, un pointeur vers un entier qui correspond à la position du minimum dans le tableau et un autre pointeur vers un entier qui correspond à la position du maximum dans le tableau. Cette fonction va retourner 3 éléments : la position du minimum et du maximum via un passage par adresse et également la somme entre le minmum et le maximum inclus. Il est important de noter que le minimum n'est pas forcément avant le maximum et également que la fonction ne doit contenir qu'une seule boucle de type "while". Par conséquent, la complexité de la fonction doit être en O(N). Nous allons donc parcourir, point par point, chaque étape au moyen d'une approche constructive.
%%%%%%%%%%%%%%%%%%%%%%%
