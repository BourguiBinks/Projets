% !TEX root = ./main.tex
%%%%%%%%%%%%%%%%%%%%%%%%%%%%%%%%%%%%%%%%%%%%%%%%%%%%%%%%%%%%%%%%%%%%%%%%%%%%%%%%%%%%%%%%%%
% Dans ce fichier, vous devez définir (Input/Output/O.U.) proprement et clairement le    %
% problème.
%
% Il est aussi demandé de réaliser une analyse complète (i.e., découpe en SPs)           %
%%%%%%%%%%%%%%%%%%%%%%%%%%%%%%%%%%%%%%%%%%%%%%%%%%%%%%%%%%%%%%%%%%%%%%%%%%%%%%%%%%%%%%%%%%

\section{Définition et Analyse du Problème}\label{analyse}
%%%%%%%%%%%%%%%%%%%%%%%%%%%%%%%%%%%%%%%%%%%%
\subsection{Input}
\paragraph{-$T$, tableau d'entiers \\ 
-$N$, taille du tableau\\
-$*min\_pos$, un pointeur vers la position du minimum\\
-$*max\_pos$, un pointeur vers la position du maximum}
\subsection{Output}
\paragraph{La somme des éléments entre la position minimale et la position maximale (inclus)}
\subsection{Objets Utilisés}
\paragraph{-$T$, tableau d'entiers ($int *T$) \\
-$N$, valeur entière ($int\;N$)\\
-$sum$, valeur entière ($int \;sum$)\\
-$new\_sum$, valeur entière ($int\; new\_sum$)\\
}
\subsection{Découpe en sous problèmes}
\paragraph{SP 1: calcul de MAXPOS(T,0,i) et de MINPOS(T, 0, i)\\
SP 2: Mettre à jour l'indice de max\_pos et min\_pos respectivement par MAXPOS(T,0,i) et MINPOS(T,0,i)\\
SP 3: Calcul de SUM(T,N,min\_pos,max\_pos)\\} 
\subsection{Emboitement}
\paragraph{SP1 $\rightarrow$ (SP3 $\subset$ SP2)}





